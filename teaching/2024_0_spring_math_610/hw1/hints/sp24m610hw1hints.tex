\documentclass{article}
\usepackage{amsmath, amsthm, amssymb, mathtools}

\mathtoolsset{showonlyrefs}

\theoremstyle{definition}
\newtheorem{definition}{Definition}

\theoremstyle{plain}
\newtheorem{proposition}{Proposition}
\newtheorem{corollary}{Corollary}
\newtheorem{theorem}{Theorem}

\newcommand{\R}{\mathbb R}
\renewcommand{\d}{\mathrm d}

\DeclareMathOperator*{\esssup}{ess\ sup}

\title{MATH 610 Homework 1 Hints}
\author{Jordan Hoffart}
\date{\today}

\begin{document}

\maketitle

\section{Exercise 1}

\subsection{Problem 1}
A function $u$ belongs to $H^1(-1,1)$ if and only if
\begin{enumerate}
  \item $u$ belongs to $L^2(-1,1)$,
  \item $u$ has a weak derivative in $L^2(-1,1)$.
\end{enumerate}
A function $u$ belongs to $L^2(-1,1)$ if and only if the integral 
\begin{equation}
  \int_{-1}^1u(x)^2\,\d x
\end{equation}
exists and is finite.
One way to show this is to explicitly compute the integral.
There is a more elegant way to do this without computing anything.
I'll let you figure that one out.

To find a weak derivative of $u$, let $\varphi$ be a test function, meaning that $\varphi \in C_0^\infty([-1,1])$, which means that 
\begin{enumerate}
  \item $\varphi$ is infinitely differentiable,
  \item $\varphi(-1) = 0$,
  \item $\varphi(1) = 0$.
\end{enumerate}
Then split up the integral over the pieces of $u$
\begin{equation}
  \int_{-1}^1u(x)\varphi'(x)\,\d x = \int_{-1}^0u(x)\varphi'(x)\,\d x + \int_0^1u(x)\varphi'(x)\,\d x
\end{equation}
and then do integration by parts.
See what falls out at the end to find a candidate for the weak derivative $v$ of $u$, and then check if $v \in L^2(-1,1)$.

\subsection{Problem 2}
This is a generalization of problem 1, so we proceed similarly.
This time, explicitly compute the integral of $u(x)^2$ where $u(x) = |x|^{\alpha}$ and see which values of $\alpha$ give you a finite integral.
This tells you which $\alpha$ allows $u \in L^2(a,b)$.
Then let $\varphi$ be a test function, do integration by parts over the pieces of $u$ as in problem 1, and see what falls out to give you a candidate for the weak $v$ derivative of $u$.
This will tell you which $\alpha$ allows for the integration-by-parts to happen at all and thus give you a candidate.
Then compute the integral of $v(x)^2$ to see which $\alpha$ allow for $v \in L^2(-1,1)$.

\subsection{Problem 3}
Be careful here.
Observe that if $\varphi$ is a test function on $[-1,1]$, this does not mean that the restriction $\varphi|_{[-1,0]}$ is a test function on $[-1,0]$, nor is it a test function when restricted to $[0,1]$.
Therefore, we cannot immediately do the following calculation
\begin{align*}
  \int_{-1}^1u(x)\varphi'(x)\,\d x &= \int_{-1}^0u_1(x)\varphi'(x)\,\d x + \int_0^1u_2(x)\varphi'(x)\,\d x \\
                                   & = -\int_{-1}^0u_1'(x)\varphi(x)\,\d x - \int_0^1u_2'(x)\varphi(x)\,\d x
\end{align*}
since we can only go to the second line when $\varphi|_{[-1,0]}$ is a test function on $[-1,0]$ and $\varphi|_{[0,1]}$ is a test function on $[0,1]$.
We have to be a bit more clever here to justify this.
To start, we need the following theorem.

\begin{theorem}
  If $u \in H^1(a,b)$, then there is a sequence $u_n$ of smooth functions in $C^{\infty}(a,b)$ that converges to $u$ in $H^1(a,b)$.
\end{theorem}

We apply this to $u_1$ and $u_2$ to get a sequence $u_n^1$ in $H^1(-1,0)$ and a sequence $u_n^2$ in $H^1(0,1)$ where $u_n^i$ converges to $u_i$.
Let 
\begin{equation}
  u_n(x) = \begin{cases}
    u_n^1(x) & x \in (-1,0) \\
    u_n^2(x) & x \in [0,1)
  \end{cases}
\end{equation}
Then we have that 
\begin{align}
  \int_{-1}^1u_n(x)\varphi'(x)\,\d x &= \int_{-1}^0u_n^1(x)\varphi'(x)\,\d x + \int_0^1u_n^2(x)\varphi'(x)\,\d x \\
                                     &= -\int_{-1}^0 (u_n^1)'(x)\varphi(x)\,\d x - \int_0^1(u_n^2)'(x)\varphi(x)\,\d x + (u_n^1(0) - u_n^2(0))\varphi(0) \\
                                     &= - \int_{-1}^0 v_n(x)\varphi(x)\,\d x + (u_n^1(0) - u_n^2(0))\varphi(0)
\end{align}
where the second step follows from integration-by-parts, which is allowed now because the $u_n^i$ are classically smooth and not only in $H^1$, and 
\begin{equation}
  v_n(x) = \begin{cases}
    (u_n^1)'(x) & x \in (-1,0) \\
    (u_n^2)'(x) & x \in [0,1)
  \end{cases}.
\end{equation}
Set 
\begin{equation}
  v(x) = \begin{cases}
    u_1'(x) & x \in (-1,0) \\
    u_2'(x) & x \in [0,1)
  \end{cases}.
\end{equation}
Show that 
\begin{align}
  \int_{-1}^1u_n(x)\varphi'(x)\,\d x & \to \int_{-1}^1 u(x) \varphi'(x)\,\d x, \\
  \int_{-1}^1v_n(x)\varphi(x)\,\d x & \to \int_{-1}^1 v(x)\varphi(x)\,\d x, \\
  (u_n^1(0) - u_n^2(0))\varphi(0) & \to 0, \\
  v & \in L^2(-1,1)
\end{align}
as $n \to \infty$ and conclude that $u \in H^1(-1,1)$ with $v$ as its weak derivative.
You are free to use the following theorem, which is a consequence of the optional problem 5.

\begin{theorem}
  Fix $x_0 \in [a,b]$. 
  Then the map
  \begin{equation}
    E_{x_0}(u) = u(x_0)
  \end{equation}
  is a continuous linear functional on $H^1(a,b)$.
  In other words,
  \begin{equation}
    E_{x_0}(cu+v) = cE_{x_0}(u) + E_{x_0}(v)
  \end{equation}
  for all $u,v \in H^1(a,b)$ and all $c \in \R$, and there is a constant $C > 0$ such that 
  \begin{equation}
    |E_{x_0}(u)| \leq C\|u\|_{H^1(a,b)}
  \end{equation}
  for all $u \in H^1(a,b)$.
\end{theorem}

\subsection{Problem 4}
Let $u \in H^1(a,b)$.
Then there is a sequence $v_n$ of smooth functions in $C^{\infty}(a,b)$ that converges to $u$ in $H^1(a,b)$.
Use the fact that
\begin{equation}
  \|v\|_{L^{\infty}(a,b)} \leq C\|v\|_{H^1(a,b)}
\end{equation}
when $v \in C^\infty(a,b)$ (which is a consequence of the last inequality in question 2 of exercise 2) to argue that $v_n$ is Cauchy in $L^{\infty}(a,b)$.
Since $L^{\infty}(a,b)$ is complete, this implies that there exists $v \in L^{\infty}(a,b)$ such that $v_n \to v$ in $L^{\infty}(a,b)$. 
Now argue that $u = v$.

\section{Exercise 2}
\subsection{Problem 1}
Use the triangle inequality and the Cauchy-Schwarz inequality
\begin{align}
  |u(x)| & \leq |u(y)| + \int_0^1|u'(s)|\,\d s \\
         & = |u(y)| + (1, |u'|)_{L^2(0,1)} \\
         & \leq |u(y)| + \|u'\|_{L^2(0,1)}.
\end{align}
Now pick particular points for $y$.

\subsection{Problem 2}
For the first inequality, first integrate 
\begin{equation}
  u(x) = \int_0^1u(y)\,\d y + \int_0^1\int_y^xu'(s)\,\d s\,\d y.
\end{equation}
Then  use the triangle inequality and Cauchy-Schwarz:
\begin{equation}
  |u(x)| \leq |\overline u| + \|u'\|_{L^2(0,1)}
\end{equation}
where 
\begin{equation}
  \overline u = \int_0^1u(y)\,\d y.
\end{equation}
At some point you will need to use Young's inequality:
\begin{equation}
  2ab \leq a^2 + b^2,
\end{equation}
which follows from the fact that 
\begin{equation}
  (a-b)^2 \geq 0
\end{equation}
for all $a, b \in \R$.

For the second and third inequalities, do essentially the same thing as problem 1, pick particular points for $y$, and apply Young's inequality.

For the last inequality, proceed as in problem 1 to get
\begin{equation}
  |u(x)| \leq |u(y)| + \|u'\|_{L^2(0,1)}.
\end{equation}
Then square, apply Young's inequality, and integrate.


\end{document}
