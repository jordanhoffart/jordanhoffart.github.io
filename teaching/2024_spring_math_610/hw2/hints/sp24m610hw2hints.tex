\documentclass{article}
\usepackage{amsmath, amsthm, amssymb, mathtools}

\mathtoolsset{showonlyrefs}

\theoremstyle{definition}
\newtheorem{definition}{Definition}

\theoremstyle{plain}
\newtheorem{proposition}{Proposition}
\newtheorem{corollary}{Corollary}
\newtheorem{theorem}{Theorem}

\newcommand{\R}{\mathbb R}
\renewcommand{\d}{\mathrm d}

\DeclareMathOperator*{\esssup}{ess\ sup}

\title{MATH 610 Homework 2 Hints}
\author{Jordan Hoffart}
\date{\today}

\begin{document}

\maketitle

\section{Exercise 1}

\subsection{Problem 1}
Suppose that you have a smooth function $u$ that satisfies the boundary conditions $u(0) = u(1) = 0$ and which solves the ODE
\begin{equation}
    -(ku')' + bu' + qu = f
\end{equation}
on $(0,1)$.
Let $v$ be another smooth function that also satisfies the boundary conditions.
Multiply the ODE by $v$ and integrate by parts to arrive at an expression of the form
\begin{equation}
    a(u,v) = F(v)
\end{equation}
where $a(u,v)$ involves integrals with $u',v',u,v,k,b$, and $q$, and $F(v)$ involves an integral with $f$ and $v$.
Now determine what Sobolev space $V$ that $u$ and $v$ should belong to so that the bilinear form $a:V\times V \to \mathbb R$ and the linear form $F:V\to\mathbb R$ are well-defined, \emph{and} which also incorporates the boundary conditions.
The weak formulation is the following problem: find $u \in V$ such that 
\begin{equation}
    a(u,v) = F(v)
\end{equation}
for all $v \in V$.

\subsection{Problem 2}
Such stability estimates are also called a priori (a Latin phrase meaning ``from before") estimates.
They are called such estimates because they are done \emph{before} we actually know if we have a solution to the ODE.
They always start in the following way: suppose that we have a solution $u \in V$ (where $V$ is chosen in problem 1) such that 
\begin{equation}
    a(u,v) = F(v)
\end{equation}
for all $v \in V$ (where $a$ and $F$ are also from problem 1).
If you chose $F$ correctly, you should be able to show that 
\begin{equation}
    F(v) \leq \|f\|_{L^2(0,1)}\|v\|_{L^2(0,1)}
\end{equation}
for all $v \in V$.
If you chose $a$ correctly, you should be able to show that 
\begin{equation}
    a(u,u) \geq \overline k \|u'\|_{L^2(0,1)}^2
\end{equation}
for all $u \in V$.
The last ingredient you will need is the following Poincar\'e inequality:
\begin{theorem}
    Let $x_0 \in [a,b]$ and let $H^1_{x_0}(a,b)$ be the space of all functions $u \in H^1(a,b)$ such that $u(x_0) = 0$.
    Then there is a constant $C > 0$ such that 
    \begin{equation}
        \|u\|_{L^2(a,b)} \leq C\|u'\|_{L^2(a,b)}.
    \end{equation}
\end{theorem}
\begin{proof}
    If $u$ is a smooth function such that $u(x_0) = 0$, then for any $x > x_0$ we have that 
    \begin{equation}
        u(x) = \int_{x_0}^xu'(t)\,\d t.
    \end{equation}
   Therefore, by Cauchy-Schwarz,
   \begin{align}
       |u(x)| \leq \int_{x_0}^x|u'(t)|\,\d t \leq \sqrt{b-a}\|u'\|_{L^2(a,b)}.
   \end{align}
   Now for $x < x_0$, we have that
   \begin{equation}
       u(x) = -\int_x^{x_0}u'(t)\,\d t,
   \end{equation}
   so we can repeat a similar argument to conclude that
   \begin{equation}
       |u(x)| \leq \sqrt{b-a}\|u'\|_{L^2(a,b)}
   \end{equation}
   for all $x \in [a,b]$.
   This implies that 
   \begin{equation}
       \|u\|_{L^2(a,b)} \leq (b-a)\|u'\|_{L^2(a,b)}
   \end{equation}
   for all smooth functions $u$ such that $u(x_0) = 0$.
   
   Now let $u \in H^1_{x_0}(a,b)$.
   Then since smooth functions that vanish at $x_0$ are dense in $H^1_{x_0}(a,b)$, there is a sequence $(u_n)_n$ of smooth functions that vanish at $x_0$ such that $\|u-u_n\|_{H^1(a,b)} \to 0$ as $n\to\infty$.
   Then $\|u-u_n\|_{L^2(a,b)} \to 0$ as $n \to \infty$ and $\|u_n'\|_{L^2(a,b)} \to \|u'\|_{L^2(a,b)}$ as $n \to \infty$.
   Then for each $n$,
   \begin{align}
       \|u\|_{L^2(a,b)} & \leq \|u_n\|_{L^2(a,b)} + \|u-u_n\|_{L^2(a,b)} \\ & \leq (b-a)\|u_n'\|_{L^2(a,b)} + \|u-u_n\|_{L^2(a,b)} \to (b-a)\|u'\|_{L^2(a,b)}
   \end{align}
   as $n\to\infty$.
   This finishes the proof.
\end{proof}
Combining everything together will give you the stability result.

\section{Exercise 2}
\subsection{Problem 1}
\subsubsection{Part a}
Multiply by a test function and integrate by parts.
The boundary condition at $x=1$ is something we have seen before, but now for the boundary condition at $0$, use it to substitute for $u'(0)$.
Rearrange everything and you will get something of the form
\begin{equation}
    a(u,v) = F(v)
\end{equation}
where $a(u,v)$ involves integrals with $u',v'$ as well as values $u(0),v(0)$ and $\beta$, while $F(v)$ will involve an integral with $f,v$ as well as the values $v(0),\gamma$, and $\beta$.
Once again, look at the bilinear form $a$ and the linear form $F$ to decide which Sobolev space the functions $u,v$ should belong to for the values $a(u,v)$ and $F(v)$ to be well-defined and to also incorporate the boundary conditions from the problem.
Hint: you already included the boundary condition at $0$ in a weak sense when you did the substitution, but now what about the boundary condition at $x = 1$?

\subsubsection{Part b}
Check the assumptions of the Lax-Milgram Theorem, which we recall below.
\begin{theorem}
    Let $V$ be a Hilbert space with inner product $(\cdot,\cdot)_V$ and induced norm $\|v\|_V := \sqrt{(v,v)_V}$.
    Let $a:V\times V\to\mathbb R$ and $F:V\to\mathbb R$ be a bilinear form and a linear form on $V$ respectively.
    Suppose that 
    \begin{enumerate}
        \item $a$ is continuous on $V$: there exists $C > 0$ such that 
        \begin{equation}
            |a(u,v)| \leq C\|u\|_V\|v\|_V
        \end{equation}
        for all $v \in V$
        \item $F$ is continuous on $V$: there exists $C' > 0$ such that 
        \begin{equation}
            |F(v)| \leq C'\|v\|_V
        \end{equation}
        for all $v \in V$
        \item $a$ is coercive (also known as elliptic) on $V$: there exists $\alpha > 0$ such that 
        \begin{equation}
            a(u,u) \geq \alpha \|u\|_V^2
        \end{equation}
        for all $u \in V$
    \end{enumerate}
    Then there is a unique $u \in V$ such that 
    \begin{equation}
        a(u,v) = F(v)
    \end{equation}
    for all $v\in V$.
\end{theorem}
If you chose $a$, $V$, and $F$ correctly in part a, you will be able to verify all of these assumptions.
For the continuity assumptions, you will need the following, which is a corollary from some of the results in your last homework.
\begin{theorem}
    There is a constant $C$ such that 
    \begin{equation}
        |u(x)| \leq C\|u\|_{H^1(a,b)}
    \end{equation}
    for all $x \in [a,b]$ and all $u \in H^1(a,b)$.
\end{theorem}
For coercivity, you will need to use the Poincar\'e inequality that I showed earlier.

\subsubsection{Part c}
You can show either an estimate of the form
\begin{equation}
    \|u\|_{H^1(0,1)} \leq E(f, \gamma, \beta)
\end{equation}
or 
\begin{equation}
    \|u'\|_{L^2(0,1)} \leq \widetilde E(f,\gamma,\beta)
\end{equation}
where $u$ is the solution to the weak problem that we showed exists from part b and $E(f,\gamma,\beta)$ and $\widetilde E(f,\gamma,\beta)$ are some continuous expressions involving the function $f$ and the boundary data $\gamma$ and $\beta$.
By the Poincar\'e inequality, we have that 
\begin{equation}
    \|u'\|_{L^2(0,1)} \leq \|u\|_{H^1(0,1)} \leq C\|u'\|_{L^2(0,1)}
\end{equation}
so that the inequalities above are equivalent: one holds for some $E$ iff the other holds for some $\widetilde E$.
The argument is similar to stuff we have done earlier in the homework: you will have to use the coercivity of $a$, the continuity of $F$, and possibly the Poincar\'e inequality.
Also, you cannot simply cite Lax-Milgram in this problem since it asks you to derive it yourself.

\subsubsection{Part d}
If $a(u_1,v) = F(v) = a(u_2,v)$ for all $v \in V$, then
\begin{equation}
    a(u_1,v) - a(u_2,v) = 0
\end{equation}
for all $v \in V$.
Now use bilinearity and coercivity.

\subsection{Problem 2}
\subsubsection{Part a}
Suppose $u$ and $v$ are smooth, undo the integration by parts and use the boundary condition $u(1) = 0$ to get something of the form
\begin{equation}
    \int_0^1 (Du - f)v\,\d x + (\text{boundary term at }x = 0) = 0 
\end{equation}
for all smooth $v$ (and, by density, all $v \in V$), where $Du$ is some expression involving $u'',\alpha,$ and $u$.
Since $V$ contains functions that vanish at $x = 0$, argue that this implies 
\begin{align}
    \int_0^1(Du-f)v\,\d x & = 0 \text{ for all } v \in C_c^\infty(0,1) \\
    (\text{boundary term at }x=0) & = 0 \text{ for all } v \in V
\end{align}
The hint in the homework tells you what ODE $u$ satisfies on $(0,1)$, while picking $v$ to be a smooth function that does not vanish at $x = 0$ in the boundary term equation will give you another boundary term that $u$ must satisfy at $x = 0$.
Therefore, your answer should be of the form
\begin{align}
    \text{ODE that } u \text{ satisfies } & \text{ on } (0,1) \\
    \text{boundary condition at } x = 0 \\
    \text{boundary condition at } x = 1
\end{align}

\subsubsection{Part b}
Same routine as the last energy estimates: use coercivity of the left side, continuity of the right side, and maybe a Poincar\'e inequality depending on if you're estimating $\|u\|_{H^1(0,1)}$ or $\|u'\|_{L^2(0,1)}$.
\end{document}
