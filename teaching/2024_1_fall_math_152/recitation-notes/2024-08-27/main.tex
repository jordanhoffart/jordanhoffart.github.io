\documentclass{article}

\title{Recitation notes}
\author{Jordan Hoffart}
\date{August 27, 2024}

\usepackage{amsmath,amsthm,amssymb,graphicx,hyperref}

\theoremstyle{plain}
\newtheorem{theorem}{Theorem}

\theoremstyle{definition}
\newtheorem{definition}{Definition}
\newtheorem{example}{Example}

\begin{document}
\maketitle
\begin{enumerate}
	\item Section 5.3 problems
	      \begin{enumerate}
		      \item Problem 19: Evaluate the integral
		            \begin{equation}
			            \int_1^3 x^2 + 2x - 4 \,dx
		            \end{equation}
		            \begin{proof}[Solution]
			            \begin{align}
				            \int_1^3 x^2 + 2x - 4 \,dx & = (\frac{1}{3}x^3 + x^2 - 4x)\bigg|_1^3          \\
				                                       & = \frac{1}{3}27 + 9 - 12 - (\frac{1}{3} + 1 - 4) \\
				                                       & = 9 - \frac{1}{3}                                \\
				                                       & = \frac{26}{3}
			            \end{align}
		            \end{proof}
		      \item Problem 39: Evaluate the integral
		            \begin{equation}
			            \int_{1/\sqrt{3}}^{\sqrt{3}} \frac{8}{1+x^2}\,dx
		            \end{equation}
		            \begin{proof}[Solution]
			            \begin{align}
				            \int_{1/\sqrt{3}}^{\sqrt{3}} \frac{8}{1+x^2}\,dx & = 8 \arctan x \bigg|_{1/\sqrt{3}}^{\sqrt{3}} \\
				                                                             & = 8 (\pi / 3 - \pi / 6)                      \\
				                                                             & = 4 \pi / 3.
			            \end{align}
		            \end{proof}
	      \end{enumerate}
	\item Section 5.5 Problems
	      \begin{enumerate}
		      \item Problem 15: Compute the indefinite integral with $u$ substituion
		            \begin{equation}
			            \int \cos^3\theta \sin \theta \,d\theta
		            \end{equation}
		            \begin{proof}[Solution]
			            Let $u = \cos \theta$.
			            Then
			            \begin{align}
				            \int \cos^3\theta \sin \theta \,d\theta & = -\int u^3\,du                 \\
				                                                    & = -\frac{1}{4}u^4 + C           \\
				                                                    & = -\frac{1}{4}\cos^4\theta + C.
			            \end{align}
		            \end{proof}
		      \item Problem 21: Compute the indefinite integral by using $u$ substitution
		            \begin{equation}
			            \int \frac{(\ln x)^2}{x}\,dx
		            \end{equation}
		            \begin{proof}[Solution]
			            Let $u = \ln x$.
			            Then
			            \begin{align}
				            \int \frac{(\ln x)^2}{x}\,dx & = \int u^2 \, du            \\
				                                         & = \frac{1}{3}u^3 + C        \\
				                                         & = \frac{1}{3}(\ln x)^3 + C.
			            \end{align}
		            \end{proof}
	      \end{enumerate}
\end{enumerate}
\end{document}
