\documentclass{article}
\title{Recitation notes}
\author{Jordan Hoffart}
\date{August 20, 2024}
\usepackage{amsmath,amsthm,amssymb,hyperref}
\theoremstyle{plain}
\newtheorem{theorem}{Theorem}
\begin{document}
\theoremstyle{definition}
\newtheorem{definition}{Definition}
\newtheorem{example}{Example}
\maketitle
\section{Organizational matters}
\begin{enumerate}
	\item MATH 152 Sections 504/505/506.
	\item Canvas page \url{https://canvas.tamu.edu/courses/331381}.
	      % \item Syllabus \url{https://canvas.tamu.edu/courses/331381/pages/math-152-syllabus-fall-2024-2?module_item_id=10956389}.
	\item Website \url{https://jordanhoffart.github.io/teaching/f24m152}
	\item Course page \url{https://www.math.tamu.edu/courses/math152/}
	\item Tuesday recitations in HEB 137/222.
	      Thursday labs in BLOC 123/124.
	\item Recitations
	      \begin{enumerate}
		      \item Review last week's material.
		      \item Take a quiz.
	      \end{enumerate}
	\item Labs
	      \begin{enumerate}
		      \item Bring your own device.
		      \item Programming in Python.
		      \item More details on Thursday.
	      \end{enumerate}
\end{enumerate}
\section{Fundamental theorem of calculus}
\begin{definition}[Antiderivative]
	Let $f$ be a function defined on an interval $I$.
	An antiderivative of $f$ is a differentiable function $F$ defined on $I$ such that
	\begin{equation}
		F'(x) = f(x)
	\end{equation}
	for all $x$ in $I$.
\end{definition}
\begin{example}
	Let $f(x) = 3x^2$.
	Then $F(x) = x^3$ is an antiderivative of $f$.
	In fact, for any constant $C$, $F(x) = x^3 + C$ is an antiderivative of $f$.
	One can show that every antiderivative of $f$ is of this form.
	That is, if $F$ is an antiderivative of $f$, then there is a constant $C$ such that $F(x) = x^3 + C$.
\end{example}
\begin{theorem}[General form of an antiderivative]
	If $F$ is an antiderivative of $f$, then for any constant $C$, $F + C$ is also an antiderivative of $f$.
	Conversely, if $F,G$ are antiderivatives of a function $f$ on an interval $I$, then there is a constant $C$ such that
	\begin{equation}
		F(x) = G(x) + C
	\end{equation}
	for all $x$ in $I$.
\end{theorem}
\begin{theorem}[Fundamental theorem of calculus, part 1]
	If $f$ is a continuous function on an interval $I$ containing a point $a$ and we define the function $F$ on $I$ by
	\begin{equation}
		F(x) = \int_a^xf(t)\,dt
	\end{equation}
	then $F$ is an antiderivative of $f$ on $I$.
\end{theorem}
\begin{example}
	Let $f(x) = e^{x^2}$ on $[0,1]$.
	Then
	\begin{equation}
		F(x) = \int_0^x e^{t^2}\,dt
	\end{equation}
	is an antiderivative of $f$, so
	\begin{equation}
		\frac{d}{dx}\int_0^x e^{t^2}\,dt = \frac{d}{dx}F(x) = f(x) = e^{x^2}.
	\end{equation}
\end{example}
\begin{theorem}[Fundamental theorem of calculus, part 2]
	If $f$ is a continuous function on an interval $[a,b]$
	and if $F$ is an antiderivative of $f$ on $[a,b]$, then
	\begin{equation}
		\int_a^b f(x)\,dx = F(b) - F(a).
	\end{equation}
\end{theorem}
\begin{example}
	Let $f(x) = \cos x$, so that $F(x) = \sin x$ is an antiderivative of $f$.
	Then
	\begin{equation}
		\int_0^\pi \cos x\, dx = \int_0^\pi f(x)\,dx = F(\pi) - F(0) = \sin \pi - \sin 0 = 0.
	\end{equation}
\end{example}
\section{Substitution rule}
\begin{definition}[Indefinite integral]
	If $f$ is a function with an antiderivative $F$, then the indefinite integral of $f$ is the collection of all antiderivatives of $f$.
	We denote this collection by
	\begin{equation}
		\int f(x)\,dx.
	\end{equation}
	The use of the letter $x$ in our notation is arbitrary.
	We can use any other symbol, as long as we are consistent.
	That is, all of these notations represent the indefinite integral of $f$:
	\begin{equation}
		\int f(x)\,dx = \int f(y)\,dy = \int f(z)\,dz = \int f(u)\,du = \dots
	\end{equation}
	as well as any other choices of the symbol of integration.

	If $F$ is an antiderivative of $f$, we will abuse notation and denote the indefinite integral of $f$ by
	\begin{equation}
		\int f(x)\,dx = F(x) + C.
	\end{equation}
\end{definition}
\begin{theorem}[Substitution rule for indefinite integrals]
	If $g$ is a differentiable function on an interval $I$, and if $f$ is a continuous function on an interval $J$ containing the range $g(I)$ of $g$, then
	$G : I \to \mathbb R$ is an antiderivative of $(f\circ g)g'$ iff $G = F \circ g$ for some antiderivative $F : J \to \mathbb R$ of $f$.
	We formally summarize this by writing
	\begin{equation}
		\int f(g(x))g'(x)\,dx = \int f(u)\,du
	\end{equation}
	with $u = g(x)$.
	We also summarize this by writing
	\begin{equation}
		\int f(g(x))g'(x)\,dx = F(g(x)) + C.
	\end{equation}
\end{theorem}
\begin{proof}
	Let $F$ be an antiderivate of $f$ and suppose $G = F\circ g$.
	Then $(F\circ g)'(x) = F'(g(x))g'(x) = f(g(x))g'(x)$, so $G = F \circ g$ is an antiderivative of $(f \circ g)g'$.

	Now let $G$ be an antiderivative of $(f\circ g)g'$.
	We pick a point $a \in J$ and let
	\begin{equation}
		\widetilde F(x) = \int_a^x f(t)\,dt.
	\end{equation}
	Then we have that $\widetilde F$ is an antiderivative of $f$.
	From above, we then have that $\widetilde F \circ g$ is an antiderivative of $(f\circ g)g'$.
	Since $G$ is also an antiderivative of $(f\circ g)g'$, there is a constant $C$ such that
	\begin{equation}
		G(x) = \widetilde F(g(x)) + C
	\end{equation}
	for all $x$ in $I$.
	Then we let
	\begin{equation}
		F(x) = \widetilde F(x) + C,
	\end{equation}
	so that $F$ is an antiderivative of $f$ for which $G = F \circ g$.
\end{proof}
\begin{example}
	Let $f(x) = \cos x$ and $g(x) = x^3$ on $\mathbb R$.
	Then by formally setting $u = g(x)$,
	\begin{align}
		\int \cos (x^3) 3x^2\,dx & = \int f(g(x))g'(x)\,dx \\
		                         & = \int f(u)\,du         \\
		                         & = \int \cos u \,du      \\
		                         & = \sin u + C            \\
		                         & = \sin x^3 + C.
	\end{align}
\end{example}
\begin{theorem}[Substitution rule for definite integrals]
	If $g$ is a differentiable function on an interval $[a,b]$, and if $f$ is a continuous function on the interval between $g(a)$ and $g(b)$, then
	\begin{equation}
		\int_a^b f(g(x))g'(x)\,dx = \int_{g(a)}^{g(b)}f(u)\,du.
	\end{equation}
\end{theorem}
\begin{example}
	Let $g(x) = \sin x$ on $[0,\pi]$ and let $f(x) = x^2$.
	Then
	\begin{align}
		\int_0^\pi \sin(x)^2\cos x\,dx & = \int_a^b f(g(x))g'(x)\,dx  \\
		                               & = \int_{g(a)}^{g(b)}f(u)\,du \\
		                               & = \int_0^0 u^2\,du           \\
		                               & = 0.
	\end{align}
\end{example}
\end{document}
