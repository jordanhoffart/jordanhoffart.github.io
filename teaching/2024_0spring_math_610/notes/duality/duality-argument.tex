\documentclass{article}

\title{A duality argument}
\author{Jordan Hoffart}
\date{}

\usepackage{amsmath,amsthm,amssymb}

\theoremstyle{plain}
\newtheorem{theorem}{Theorem}
\newtheorem{lemma}{Lemma}

\begin{document}
\maketitle

We consider the following variational problem: find $u \in H^1(\Omega)$ such that
\begin{equation}\label{eq:1}
	a(u,v) = (f,v)
\end{equation}
for all $v \in H^1(\Omega)$.
Here, $f \in L^2(\Omega)$, $(\cdot,\cdot)$ denotes the $L^2$ inner product, and $a$ is assumed to be a continuous and coercive bilinear form on $H^1(\Omega)$.
By Lax-Milgram, we know that there is a unique $u \in H^1(\Omega)$ that solves the above problem.

For each $h > 0$, we let $V_h \subset H^1(\Omega)$ be finite dimensional.
Then we know that, for each $h > 0$, there is a unique $u_h \in V_h$ such that
\begin{equation}\label{eq:2}
	a(u_h, v_h) = (f,v_h)
\end{equation}
for all $v_h \in V_h$.

We wish to estimate the error $u-u_h$ in the $L^2$ norm.
To do this, we first estimate the error in the $H^1$ norm.
We will make assumptions needed along the way, and then we will summarize all of our results at the end.

First, we observe that we have Galerkin orthogonality:
\begin{equation}\label{eq:3}
	a(u-u_h,v_h) = 0
\end{equation}
for all $v_h \in V_h$.
Indeed, we just take $v = v_h$ in equation \eqref{eq:1} and then subtract equation \eqref{eq:2}.

In everything that follows, we use the symbol $C$ to denote a general constant whose particular value may change from line to line but does not depend on anything else involved in the estimates.
Now we use coercivity, Galerkin orthogonality, and continuity to derive the following bound:
\begin{align*}
	\|u-u_h\|_1^2 & \leq Ca(u-u_h,u-u_h)          \\
	              & = Ca(u-u_h,u-v_h)             \\
	              & \leq C\|u-u_h\|_1\|u-v_h\|_1.
\end{align*}
Here, $\|\cdot\|_1$ denotes the $H^1$ norm.
Now, since $v_h$ is arbitrary, we have that
\begin{equation}\label{eq:4}
	\|u-u_h\|_1 \leq C\inf_{v_h\in V_h}\|u-v_h\|_1.
\end{equation}
This result is known as Ce\'a's Lemma in the literature.
The constant $C$ only depends on the continuity and coercivity of the bilinear form $a$, but not on $u$ or $u_h$.
We restate it below.

\begin{lemma}[C\'ea's Lemma]
	Let $a$ be a continuous and coercive bilinear form on $H^1(\Omega)$.
	Let $f \in L^2(\Omega)$.
	Let $u \in H^1(\Omega)$ be the unique solution to
	\[a(u,v) = (f,v)\]
	for all $v\in H^1(\Omega)$.
	Let $h > 0$, let $V_h \subset H^1(\Omega)$ be finite dimensional, and let $u_h \in V_h$ be the unique solution to
	\[a(u_h,v_h) = (f,v_h)\]
	for all $v_h \in V_h$.
	Then for all $h > 0$,
	\begin{equation}
		\|u-u_h\|_1 \leq C\inf_{v_h\in V_h}\|u-v_h\|_1.
	\end{equation}
\end{lemma}

Now we assume that $u$ has the following regularity and approximation properties: $u \in H^2(\Omega)$ and
\begin{equation}\label{eq:8}
	\inf_{v_h\in V_h}\|u-v_h\|_1 \leq Ch\|u\|_2.
\end{equation}
Here, $\|\cdot\|_2$ denotes the $H^2$ norm and $C$ does not depend on $v$ or $h$, but may depend on the source term $f$.
Combining this with \eqref{eq:4} gives us
\begin{equation}
	\|u-u_h\|_1 \leq Ch\|u\|_2.
\end{equation}
This gives us an estimate on the error between the solution $u$ to \eqref{eq:1} and the solution $u_h$ to \eqref{eq:2} in the $H^1$ norm.
We summarize our results below.
\begin{theorem}[$H^1$ error]
	Let $a$ be a continuous and coercive bilinear form on $H^1(\Omega)$.
	Let $f \in L^2(\Omega)$.
	Let $u \in H^1(\Omega)$ be the unique solution to
	\[a(u,v) = (f,v)\]
	for all $v\in H^1(\Omega)$.
	Let $h > 0$, let $V_h \subset H^1(\Omega)$ be finite dimensional, and let $u_h \in V_h$ be the unique solution to
	\[a(u_h,v_h) = (f,v_h)\]
	for all $v_h \in V_h$.
	Assume that
	\begin{enumerate}
		\item $u \in H^2(\Omega)$,
		\item there is a constant $C$ such that, for all $h > 0$,
		      \[\inf_{v_h\in V_h}\|u-v_h\|_1 \leq Ch\|u\|_2.\]
	\end{enumerate}
	Then there is a constant $C > 0$ such that, for all $h > 0$,
	\begin{equation}\label{eq:6}
		\|u-u_h\|_1 \leq Ch\|u\|_2.
	\end{equation}
\end{theorem}

Now we wish to invoke a duality argument to derive an estimate of the $L^2$ error.
For that, we examine the dual problem to \eqref{eq:1}: given $g \in L^2(\Omega)$, find $w \in H^1(\Omega)$ such that
\begin{equation}\label{eq:5}
	a(v,w) = (g,v)
\end{equation}
for all $v \in H^1(\Omega)$.
We notice that the solution $w$ that we seek is now in the second argument of $a$ instead of the first like it is in \eqref{eq:1}.
If $a$ is symmetric, then \eqref{eq:5} is exactly \eqref{eq:1}.
Notice, however, that we did not assume that $a$ was symmetric.
Indeed, for the purposes of a duality argument, we do not need to.
All of what we said so far applies to non-symmetric bilinear forms.
Therefore, once again by Lax-Milgram, for any $g \in L^2(\Omega)$, there is a unique $w = w_g$ that solves \eqref{eq:5}.

We apply this to $g = u-u_h$ and test with $v = u-u_h$ to get that
\begin{equation}
	\|u-u_h\|^2 = (u-u_h,u-u_h) = (g,v) = a(v,w_g) = a(u-u_h,w_g).
\end{equation}
Here, $\|\cdot\|$ denotes the $L^2$ norm.
Now we can use Galerkin orthogonality and continuity of $a$ to get that
\begin{equation}
	\|u-u_h\|^2 = a(u-u_h, w_g-w_h) \leq C\|u-u_h\|_1\|w_g-w_h\|_1
\end{equation}
for all $w_h \in V_h$.
Since this holds for all $w_h \in V_h$, we have that
\begin{equation}
	\|u-u_h\|^2 = a(u-u_h, w_g-w_h) \leq C\|u-u_h\|_1\inf_{w_h\in V_h}\|w_g-w_h\|_1.
\end{equation}
We can use the error estimate \eqref{eq:6} above to get that
\begin{equation}\label{eq:7}
	\|u-u_h\|^2 \leq Ch\|u\|_2\inf_{w_h\in V_h}\|w_g-w_h\|_1.
\end{equation}
Now we make the following regularity and approximation assumption about solutions to the dual problem: for every $g \in L^2(\Omega)$, the solution to the dual problem \eqref{eq:5} $w_g \in H^2(\Omega)$ and
\begin{equation}
	\inf_{w_h \in V_h}\|w_g-w_h\|_1 \leq Ch\|w_g\|_2.
\end{equation}
Here, unlike the assumption \eqref{eq:8}, the constant $C$ is independent of $h$, $w_g$, and $g$.
Using this regularity assumption in \eqref{eq:7} lets us conclude that
\begin{equation}
	\|u-u_h\|^2 \leq Ch^2\|u\|_2\|w_g\|_2.
\end{equation}
Now we make one last assumption: there is a constant $C$ such that for all $g \in L^2(\Omega)$,
\begin{equation}
	\|w_g\|_2 \leq C\|g\|.
\end{equation}
Applying this and remembering that $g = u-u_h$ gives us
\begin{equation}
	\|u-u_h\|^2 \leq Ch^2\|u\|_2\|g\| = Ch^2\|u\|_2\|u-u_h\|.
\end{equation}
We now summarize all of our assumptions and our conclusions.

\begin{theorem}[Duality argument]
	Let $a$ be a continuous and coercive bilinear form on $H^1(\Omega)$.
	Let $f \in L^2(\Omega)$.
	Let $u \in H^1(\Omega)$ be the unique solution to
	\[a(u,v) = (f,v)\]
	for all $v\in H^1(\Omega)$.
	Let $h > 0$, let $V_h \subset H^1(\Omega)$ be finite dimensional, and let $u_h \in V_h$ be the unique solution to
	\[a(u_h,v_h) = (f,v_h)\]
	for all $v_h \in V_h$.
	For each $g \in L^2(\Omega)$, let $w_g \in H^1(\Omega)$ be the unique solution to the dual problem
	\[a(v,w_g) = (g,v)\]
	for all $v \in H^1(\Omega)$.
	Assume that
	\begin{enumerate}
		\item $u \in H^2(\Omega)$,
		\item there is a constant $C$ such that, for all $h > 0$,
		      \[\|u-u_h\|_1 \leq Ch\|u\|_2\]
		\item for each $g \in L^2(\Omega)$, $w_g \in H^2(\Omega)$,
		\item there is a constant $C$ such that, for each $g \in L^2(\Omega)$ and each $h > 0$,
		      \[\inf_{w_h \in V_h}\|w_g-w_h\|_1 \leq Ch\|w_g\|_2\]
		\item there is a constant $C$ such that for each $g \in L^2(\Omega)$,
		      \[\|w_g\|_2 \leq C\|g\|.\]
	\end{enumerate}
	Then there is a constant $C > 0$ such that for all $h > 0$
	\begin{equation}
		\|u-u_h\| \leq Ch^2\|u\|_2.
	\end{equation}
\end{theorem}

\end{document}
